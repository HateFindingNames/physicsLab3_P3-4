\chapter{Evaluation}
    %
    \section{Calibration of the Force Sensor}
        Because of a measurement misfortune for the calibration, the common way for determining the calibration factor is
        inoperable. Therefore, another method is improvised. Originally, the calibration factor \(c\) was supposed to be
        determined by computing the slope of the correlation between the ADC values and the pulling force on the strain gauge.
        Thus, the weight of the different masses would have been converted into forces by
        \begin{align}
            c=\frac{\Delta n}{\Delta F} &&\left[\frac{1}{\SI{}{mN}}\right]
        \end{align}
        Since this measurement failed, the literature value of the surface tension of distilled water was used in
        order to be able to carry out a backward calculation. For that, \cref{eq:surface tension} is transformed into the force \(F_0\):
        \begin{align*}
            F_0=\sigma \cdot 4\pi \cdot \bar{r}
        \end{align*}
        With
        \begin{align}
            F_0=F_{max}=\frac{n_{max}}{c}
            \label{eq:force}
        \end{align}
        it follows for distilled water:
        \begin{gather}
            \frac{n_{max}}{c} = \sigma_{H_2O} \cdot 4\pi \cdot \bar{r} \nonumber \\
            \Leftrightarrow \nonumber \\
            c = \frac{n_{max}}{\sigma_{H_2O} \cdot 4\pi \cdot \bar{r}}
            \label{eq:calibration factor}
        \end{gather}
        with \(\sigma_{H_2O}=\SI{72.8}{\frac{mN}{m}}\) at \SI{20}{\celsius} \cite{Eichler.2016} and \(\bar{r} = \SI{0.03}{m} \pm \SI{0.00005}{m}\)
        (approximated with the inner radius $ r_i=\SI{0.02923}{m} $ and outer radius $ r_o=\SI{0.03}{m} $)\par\medskip
        %
        The ADC result \(n_{max}\) can be read from \textsc{Du Noüy}s ring method measurements with distilled water (\cref{fig:du_nouy_method_measurement_with_distilled_water_no_1_for_calibration}).
        There are 10 values for \(n_{max}\):
        \begin{align*}
            n_{max,i}=[80321, 77600, 77085, 80827, 76744, 82423, 79322, 77319, 79543, 77873]
        \end{align*}
        The mean value and the standard deviation are
        \begin{align*}
            \bar{n}_{max}&=78906 \\
            \Delta n_{max}&=1883
        \end{align*}
        \Cref{eq:calibration factor} gives the calibration factor with
        \begin{align*}
            \boxed{c=\SI{(2875 \pm 73)}{\frac{1}{mN}}}
        \end{align*}
        The deviation of the calibration factor is calculated as follows:
        \begin{align}
            \Delta c    &=\left| \frac{\partial c}{\partial n_{max}} \right| \cdot \Delta n_{max} + \left| \frac{\partial c}{\partial \bar{r}} \right| \cdot \Delta \bar{r} \nonumber \\
                        &=\frac{\Delta n_{max}}{\sigma_{H_2O} \cdot 4\pi \cdot \bar{r}} + \frac{n_{max} \cdot \Delta \bar{r}}{\sigma_{H_2O} \cdot 4\pi \cdot \bar{r}^2} \nonumber \\
                        &=\frac{1883}{\SI{72.8}{\frac{mN}{m}} \cdot 4\pi \cdot \SI{0.03}{m}} + \frac{78906 \cdot 5 \cdot \SI{10^{-5}}{m}}{\SI{72.8}{\frac{mN}{m}} \cdot 4\pi \cdot (\SI{0.03}{m})^2} \nonumber \\
                        &=\SI{68.61}{\frac{1}{mN}}+\SI{4.79}{\frac{1}{mN}} \nonumber \\
                        &=\SI{73.4}{\frac{1}{mN}} \approx \SI{73}{\frac{1}{mN}}
        \end{align}
        \begin{figure}[h]
            \centering
            % \includegraphics[width=.9\textwidth]{scidavis/Du_Nouy_Method_Measurement_with_distilled_water_No_1_for_cal.jpg}
            \includesvg[inkscapelatex=false, width=.9\textwidth]{scidavis/Du_Nouy_Method_Measurement_with_distilled_water_No_1_for_cal}
            \caption[Measurement with \textsc{Du Noüy} ring in distilled water for the calibration (\(\vartheta \approx \SI{20}{\celsius}\))]{Measurement with \textsc{Du Noüy} ring in distilled water for the calibration (\(\vartheta \approx \SI{20}{\celsius}\)).}
            \label{fig:du_nouy_method_measurement_with_distilled_water_no_1_for_calibration}
        \end{figure}
        %
    \section{Resolution and Statistics}
        For comparison of the raw and filtered data of the ADC with regard to statistical variations, both are
        plotted as a function of the time in a stray diagram in \cref{fig:adc_result}.\par
        As it can be seen, the raw data has a greater scattering. This is also confirmed by the standard deviations, which are as follows:
        \begin{figure}[h]
            \centering
            % \includegraphics[width=.9\textwidth]{scidavis/ADC_result.jpg}
            \includesvg[inkscapelatex=false, width=.9\textwidth]{scidavis/ADC_result}
            \caption[Stray diagram with no load applied to force sensor]{Stray diagram with no load applied to force sensor.}
            \label{fig:adc_result}
        \end{figure}
        \begin{align*}
            \bar{n}_{raw}       &=-21 \qquad \varsigma_{raw}=58\\
            \bar{n}_{filtered}  &=-23 \qquad \varsigma_{filtered}=39
        \end{align*}
        Based on that, the deviation of the force to be read can be determined as
        \begin{align}
            \Delta F    &=\left| \frac{\partial F}{\partial \bar{n}_{filtered}} \right| \cdot \varsigma_{filtered} + \left| \frac{\partial F}{\partial c} \right| \cdot \Delta c \nonumber \\
                        &=\frac{1}{c} \cdot \varsigma_{filtered} + \frac{\left|\bar{n}_{filtered}\right|}{c^2} \cdot \Delta c \nonumber \\
                        &=\frac{39}{\SI{2875}{\frac{1}{mN}}} + \frac{23 \cdot \SI{73}{\frac{1}{mN}}}{(\SI{2875}{\frac{1}{mN}})^2} \cdot \SI{73}{\frac{1}{mN}} \nonumber \\
                        &=\SI{0.0135}{mN}+\SI{0.0002}{mN} \nonumber \\
                        &=\SI{13.7}{\micro N}
        \end{align}
        The effective resolution of the ADC is given by means of \cref{eq:ENOB}:
        \begin{align*}
            ENOB&=-23-\log_2(39)=-28
        \end{align*}
        Further, the histograms (\cref{fig:histogram_raw_data} and \cref{fig:histogram_filtered_data}) show that the raw
        data is more widely spread than the filtered one. The raw data histogram has a binning value of \(\approx 40\) while the
        filtered datas binning value is \(\approx 20\).\par\medskip
        \begin{figure}[h]
            \centering
            % \includegraphics[width=.9\textwidth]{scidavis/histogram_raw_data.jpg}
            \includesvg[inkscapelatex=false, width=.9\textwidth]{scidavis/histogram_raw_data}
            \caption[Histogram of raw data]{Histogram of raw data.}
            \label{fig:histogram_raw_data}
        \end{figure}
        
        \begin{figure}[h]
            \centering
            % \includegraphics[width=.9\textwidth]{scidavis/histogram_filtered_data.jpg}
            \includesvg[inkscapelatex=false, width=.9\textwidth]{scidavis/histogram_filtered_data}
            \caption[Histogram of filtered data]{Histogram of filtered data.}
            \label{fig:histogram_filtered_data}
        \end{figure}
        %
    \section{\textsc{Du Noüy} Ring Method Measurement}
        The data is being sent to the PC during the measurement procedure. The diagram representing force over time, as it
        is displayed in \cref{fig:Du_Nouy_Method_Measurement_with_distilled_water_No_1,fig:Du_Nouy_Method_Measurement_with_detergent_No_1,fig:Du_Nouy_Method_Measurement_with_more_detergent_No_1,fig:Du_Nouy_Method_Measurement_with_isopropanol_No_1}, is obtained converting the ADC data into the force by way of \cref{eq:force}. 
        %
        \subsection{Distilled Water}
            The maximum forces for distilled water are as follows:
            \begin{align*}
                F_{max,i}^{H_2O}=[27.9, 27.0, 26.8, 28.1, 26.7, 28.7, 27.6, 26.9, 27.7, 27.1] && [\SI{}{mN}]
            \end{align*}
            The forces are around the mean value
            \begin{align*}
                \bar{F}_{max}^{H_2O}=\SI{(27.5 \pm 0.7)}{mN}
            \end{align*}
            so the calculated value in \cref{eq:calculated force} is approximated well. By using \cref{eq:surface tension}
            for the surface tension of distilled water, it results:
            \begin{align*}
                \boxed{\sigma_{H_2O}^{det}=\SI{(72.95 \pm 0.15)}{\frac{mN}{m}}}
            \end{align*}
            with a deviation
            \begin{align}
                \Delta \sigma_{H_2O}^{det}  &= \left| \frac{\partial \sigma_{H_2O}^{det}}{\partial \bar{F}_{max,H_2O}} \right| \cdot \Delta F + \left| \frac{\partial \sigma_{H_2O}^{det}}{\partial \bar{r}} \right| \cdot \Delta \bar{r} \nonumber \\
                                            &= \frac{\Delta F}{4\pi \cdot \bar{r}} + \frac{\bar{F}_{max}^{H2O}\cdot \Delta \bar{r}}{4\pi \cdot \bar{r}^2} \nonumber \\
                                            &= \frac{\SI{0.01}{mN}}{4\pi \cdot \SI{0.03}{m}} + \frac{\SI{27.5}{mN}}{4\pi \cdot (\SI{0.03}{m})^2} \cdot \SI{0.00005}{m} \nonumber \\
                                            &= \SI{0.0265}{\frac{mN}{m}} + \SI{0.122}{\frac{mN}{m}} \nonumber \\
                                            &= \SI{148.5}{\frac{\micro N}{m}} \approx \SI{150}{\frac{\micro N}{m}}
            \end{align}
            The determined value of $ \SI{72.95}{\frac{mN}{m}} $ fits well with the literature value of $ \SI{72.8}{\frac{mN}{m}} $. The area of error covers the literature value.
            %
        \subsection{Detergent}
            The maximum forces for detergent are
            \begin{align*}
                F_{max,i}^{\rho_1} = [26.8, 25.8, 26.4, 23.9, 21.0, 21.6, 23.0, 22.6, 27.9, 23.8] &&[\SI{}{mN}]
            \end{align*}
            and their mean value is
            \begin{align*}
                \bar{F}_{max}^{\rho_1}=\SI{(24.3 \pm 2.3)}{mN}
            \end{align*}
            For the detergents surface tension it results:
            \begin{align}
                \boxed{\sigma_{\rho_1}=\SI{(64.46 \pm 0.13)}{\frac{mN}{m}}}
            \end{align}
            The deviation is calculated equivalently to the one of distilled water.\par\medskip
            %
            In order to be able to compare this value with the literature value, the concentration used has to be estimated.
            Due to lack of preceding measurements of the masses \(m_{detergent}\) of detergent applied to the water, the mass
            per application is approximated with \SI{0.01}{g}. Furthermore, the type of detergent was assumed to be sodium dodecyl sulfate
            as it is a commonly used detergent. Its molar mass \(M\) is \SI{288.4}{\frac{g}{mol}} \cite{sodium.dodecyl.sulfate.cas.entry.2021}.
            For the amount of the substance \(N\) it follows:
            \begin{align}
                N = \frac{m_{detergent}}{M} = \frac{\SI{0.01}{g}}{\SI{288.4}{\frac{g}{mol}}} = \SI{34.7}{\micro mol}
                \label{eq:amount of stuff}
            \end{align}
            Thus, the concentration \(\rho\) for \(V = \SI{100}{mL}\) of water is estimated to
            \begin{align}
                \rho_1 = \frac{N}{V} = \frac{\SI{34.7}{\micro mol}}{\SI{100}{mL}} \approx \SI{0.35}{\frac{mmol}{L}}
                \label{eq:concentration}
            \end{align}
            In \cite{synth.of.ACD.as.surfactant.Kumar.2015} at \(\rho_1 = \SI{0.35}{\frac{mmol}{L}}\) a surface tension of about \SI{63}{\frac{mN}{m}}
            can be read. So there is a good match between determined and literature value.
            %
        \subsection{Raising the Concentration of Detergent}
            Following \cref{eq:concentration,eq:amount of stuff}, the new concentration raises to
            \begin{equation}
                \rho_2 = \frac{2N}{V} \approx \SI{0.69}{\frac{mmol}{L}}
            \end{equation}
            Subsequent measurements lead to maximum forces of
            \begin{align*}
                F_{max,i}^{\rho_2}=[25.8, 24.9, 20.5, 20.8, 18.5, 18.8, 20.9, 26.7, 28.0, 27.6] &&[\SI{}{mN}]
            \end{align*}
            with a mean value of
            \begin{align*}
                \bar{F}_{max}^{\rho_2}=\SI{23.3}{mN}
            \end{align*}
            This calculates to a surface tension of
            \begin{align*}
                \boxed{\sigma_{\rho_2}=\SI{(61.81 \pm 0.13)}{\frac{mN}{m}}}
            \end{align*}
            The surface tension of the liquid with added detergent is a bit smaller than before. This is an expected behavior as in \cite{synth.of.ACD.as.surfactant.Kumar.2015}
            it can be seen that the surface tension decreases by increasing its concentration.

        \subsection{Isopropanol}
            With isopropanol the following maximum forces are obtained:
            \begin{align*}
                F_{max,i}^{IPA}=[8.6, 8.0, 5.9, 5.9, 6.3, 7.7, 5.6, 7.9, 7.4, 7.9] &&[\SI{}{mN}]
            \end{align*}
            They give the mean value
            \begin{align*}
                \bar{F}_{max}^{IPA}=\SI{6.3}{mN}
            \end{align*}
            The determined surface tension of isopropanol is therefore
            \begin{align*}
                \boxed{\sigma_{IPA}=\SI{(16.71 \pm 0.05)}{\frac{\milli\newton}{\metre}}}
            \end{align*}
            The deviation is also calculated as above.\par
            Both the determined and the literature value of the surface tension of isopropanol are in the same magnitude. The
            literature value is \SI{21.4}{\frac{mN}{m}} \cite{Eichler.2016} and thus the determined one is slightly
            lower. Here, the literature value is not covered by the error area as well.
